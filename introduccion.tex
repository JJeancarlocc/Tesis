\chapter*{INTRODUCCIÓN}
\addcontentsline{toc}{chapter}{INTRODUCCIÓN}	%Comando para agregar introducción al indice
La batería ha sido un dispositivo sustancial a lo largo de la historia al momento de proveer energía a diferentes aplicaciones tales como vehículos aéreos no tripulados, vehículos eléctricos, teléfonos móviles, ordenadores portátiles, entre otros. Y debido a que la tecnología avanza, el desarrollo de las baterías ha tenido que mejorar para cumplir los requerimientos de portabilidad, eficiencia de energía y miniaturización \cite{ChristopherMims}. \\

En la actualidad las baterías secundarias, más conocidas como baterías recargables, son las más populares al momento de elegir una fuente de energía. Así mismo las baterías primarias, aquellas baterías o pilas que no son recargables, están siendo desplazadas por las baterías recargables; sin embargo, se debe resaltar que son bastantes prácticas para ciertas aplicaciones específicas \cite{Buchmann2011}. \\

A pesar de que el desarrollo de las baterías ha ido mejorando en los últimos años, en la actualidad las empresas que fabrican las baterías no proveen información detallada sobre la capacidad eléctrica que contienen éstas después de su primer uso. Incluso algunas empresas como Energizer y Duracell han dejado de brindar información sobre la cantidad de mAh de las pilas alcalinas, lo que se vuelve un problema para los usuarios al momento de querer utilizar las baterías o pilas para un proyecto de laboratorio que requiera una corriente específica.\\

En esta tesis se planteará el diseño de un instrumento de medición que tendrá como principal objetivo estimar la capacidad eléctrica que pueden proveer los bancos de baterías basados en baterías tipo li-ion, como las que se emplean en teléfonos móviles y las pilas alcalinas.
\vspace*{\stretch{0.5}}
\pagebreak